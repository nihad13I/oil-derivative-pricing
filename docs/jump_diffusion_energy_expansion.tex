\documentclass[12pt]{article}
\usepackage[utf8]{inputenc}
\usepackage{amsmath, amssymb, amsfonts}
\usepackage{geometry}
\geometry{a4paper, margin=1in}

\title{Stochastic Energy Dynamics: \\ From Mean Reversion to Merton Jump-Diffusion}
\date{} % This explicitly empties the date field so \maketitle won't show it

\begin{document}

\maketitle

\section{Introduction}
Modeling energy commodities like Brent Crude requires accounting for two distinct market behaviors: long-term equilibrium (mean reversion) and sudden, discontinuous price shocks (jumps). This paper provides the mathematical derivation for transitioning from a standard Ornstein-Uhlenbeck (OU) process to a Merton Jump-Diffusion (MJD) framework.

\section{Phase 1: The Ornstein-Uhlenbeck Process}
The foundation of our model assumes that prices $S_t$ tend to revert to a long-term mean $\theta$. The continuous dynamics are described by the SDE:
\begin{equation}
    dS_t = \kappa(\theta - S_t)dt + \sigma S_t dW_t
\end{equation}
Where $dW_t$ is a standard Wiener process. Under this model, the returns are assumed to be normally distributed, represented by the probability density function:
\begin{equation}
    f(x) = \frac{1}{\sigma\sqrt{2\pi t}} e^{-\frac{(x-\mu)^2}{2\sigma^2t}}
\end{equation}
\textit{Technical Reasoning:} While the OU model captures the equilibrium of the oil market, it fails to account for the "Fat-Tails" (high kurtosis) observed during geopolitical crises.

\section{Phase 2: Merton Jump-Diffusion Expansion}
To address the inadequacy of the normal distribution in energy markets, we introduce a Poisson-driven jump component.

\subsection{The Expanded SDE}
We define the new price process by adding a jump term $dJ_t$:
\begin{equation}
    dS_t = \kappa(\theta - S_t)dt + \sigma S_t dW_t + S_t dJ_t
\end{equation}
The jump term is defined as $dJ_t = (Y-1) dN_t$, where:
\begin{itemize}
    \item $N_t$ is a Poisson process with intensity $\lambda$, such that $P(dN_t = 1) = \lambda dt$.
    \item $Y$ is the random jump magnitude, where $\ln(Y) \sim N(\mu_j, \sigma_j^2)$.
\end{itemize}

\subsection{Log-Price Transformation}
For numerical stability and to prevent negative prices in our simulation engine, we apply Itô's Lemma to the log-price $x_t = \ln S_t$. The resulting dynamics are:
\begin{equation}
    dx_t = \left[ \kappa(\ln \theta - x_t) - \frac{1}{2}\sigma^2 \right]dt + \sigma dW_t + \ln(Y) dN_t
\end{equation}

\section{Conclusion: Technical Justification}
The transition to Jump-Diffusion is necessitated by the empirical observation of "Black Swan" events in Brent Crude data. 
\begin{itemize}
    \item \textbf{Volatility Separation}: By isolating jumps via the intensity $\lambda$, we obtain a "Clean $\sigma$" for the diffusion component, preventing the overestimation of daily market noise.
    \item \textbf{Risk Management}: The MJD model provides a more accurate Value-at-Risk (VaR) by accounting for the discontinuous price "spikes" that a smooth OU process ignores.
\end{itemize}

\end{document}
